\documentclass[a4paper, 11pt]{article}
\usepackage{styles/custom}
\usepackage{styles/acm}

\title{Document}
\author{\Large Lorem Ipsum \\ \small \url{lorem.ipsum@gmail.com}}

\begin{document}

\maketitle

\begin{abstract}
  \lipsum[1]
\end{abstract}

\section{Lorem}

\lipsum[1]

\[
  \tau \subseteq \mathcal{P}(X), A \in \tau \land B \in \tau \implies A \cup B \in \tau
\]

\lipsum[3]

\subsection{Ipsum}

\lipsum[4]

\section{Ipsum}

\begin{definition}[Lorem Ipsum]
  Lorem Ipsum is Lorem Ipsum.
\end{definition}

\begin{theorem}
  Every monoidal category is equivalent to a strict one.
\end{theorem}
\begin{proof}
  ...
\end{proof}

\lipsum[5]

\subsection{Diagram}

\lipsum[9]

\begin{center}
\begin{tikzcd}
  A 
    \ar[d, "f"']
    \ar[rd, "g \circ f"]
  & \\
  B
    \ar[r, "g"'] 
  & C
\end{tikzcd}
\end{center}

\lipsum[10]

\subsection{Typing Rule}

\begin{mathpar}
  \inferrule
    {\Gamma \vdash M : A}
    {\Gamma \vdash M : A}
\end{mathpar}

We cite \cite{Streicher91}. \lipsum[6]

\bibliographystyle{apalike}
\bibliography{main}

\end{document}
